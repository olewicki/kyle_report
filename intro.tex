\section{Introduction}

%In this report, we gathered the results we generated the last few weeks. 

%First, for the first four sections, we showed 3 groups of graph each time, let me explain here where this separation comes from. Each push is identified by a \texttt{revision\_12C}, which is the revision number of the last commit in that push. To count the number of commits by push, we gathered commits from mercurial regarding the studied period. Most of the \texttt{revision\_12C} from the initial dataset (retrieved by Nourredine) was found in the data scrapted from mercurial : 98\% of the \texttt{revision\_12C} are retrieved, covering 99.8\% of the builds. 

%Hovewer, the numbers are much different when looking at the "mozilla-central" branch, where only 57\% of the \texttt{revision\_12C} are retrieved, still covering 99.4\% of the builds. The remaining 43\% have only a low number of builds and are mainly in releases 36 to 44. See Fig~\ref{build_push_COUNT}(c) to visualise that.

%Also, it must be known that an initial count of commit by push had been done by Nourredine in the first dataset, under the name \texttt{count\_elements}. This values often differ from the mercurial dataset, differ from one build to an other even if the commit didn't changed and does sometimes omit some commits or add others. I still need to check on that.

%Thus, I computed the number of commits by push using 3 differents ways :
%\begin{itemize}
%    \item using the count from the mercurial scrap and using the \texttt{count\_elements} highest value for each \texttt{revision\_12C};
%    \item using the count from the mercurial scrap and ignoring the \texttt{revision\_12C} unfound on mercurial;
%    \item using the \texttt{count\_elements} highest value for each \texttt{revision\_12C}.
%\end{itemize}

%That's why each group of graph is represented three times. 


\newpage